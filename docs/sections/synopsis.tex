\section*{Keywords}
\begin{itemize}
    \item Gas flaring emissions
    \item Dispersion modeling
    \item Meteorological conditions
    \item Oil and Gas
    \item Aermod
\end{itemize}

\section*{Search Streams, Database, No. of Hits}
\begin{enumerate}
    \item Gas flaring emissions dispersion modelling, Google Scholar, 20,000 hits
    \item Newfoundland and Labrador gas flaring dispersion AND meteorological conditions, Google Scholar, 783 hits
    \item Impact of weather on gas flaring emissions dispersion, Google Scholar, 17,700 hits
    \item Gas flaring dispersion patterns and meteorological conditions, Google Scholar, 17,800 hits
\end{enumerate}
\section*{Inclusion/Exclusion Criteria}
\begin{itemize}
    \item Newer than 2016
    \item With exact phrase oil and gas
    \item Contains Aermod
\end{itemize}

\section*{Search Streams, Filtered Literature and Nos. of Papers}
\begin{enumerate}
    \item Search stream 1, applied inclusion/exclusion criteria, reduced to 111 papers
    \item Search stream 2, applied inclusion/exclusion criteria, reduced to 5 papers
    \item Search stream 3, applied inclusion/exclusion criteria, reduced to 108 papers 
    \item Search stream 5, applied inclusion/exclusion criteria, reduced to 80 papers
\end{enumerate}

\section*{Synopsis}

\subsection*{Paper 1}
\textbf{Title:} Dispersion of gas flaring emissions in the Niger delta: Impact of prevailing meteorological conditions and flare characteristics \cite{fawole2019dispersion}.\\
\textbf{Summary:} This study investigates the relationship between meteorological conditions, flare characteristics (such as size), and their impact on the dispersion and ground-level concentrations of pollutants from gas flares in the Niger Delta. The research aims to address a significant gap in understanding the real-world effects of gas flares on air quality, through extensive data gathered from flaring activities in the oil and gas industry. Using ADMS 5 and AERMOD dispersion models, the researchers simulated pollutant dispersion under various conditions. Findings indicate that during the West African Monsoon (WAM) months (April to September), pollutants primarily disperse over inland communities, while during non-WAM months (November to March), they disperse towards both inland and coastal areas. Smaller flares and those with lower heat content resulted in higher pollutant concentrations near the flares. The study suggests mitigating local pollution by combining short stacks flaring at lower volumes to enhance plume buoyancy, reducing ground-level concentrations. This research shows the significant impact of flare characteristics and meteorology on pollution dispersion and the necessity for strategic measures to enhance air quality.
\subsection*{Paper 2}
\textbf{Title:} Methane emission rate estimate using airborne measurement at offshore oil platforms in Newfoundland and Labrador, Canada \cite{khaleghi2023methane}.\\
\textbf{Summary:} The study addresses the knowledge gap in methane (\(\CHfour\)) emissions from offshore oil production in Newfoundland and Labrador, which previously lacked targeted measurements despite low emission intensities reported by the industry. The researchers used a Twin Otter aircraft equipped with a Picarro gas analyzer and an Aventech wind system to measure \CHfour\ emissions from three offshore oil facilities. They applied two methods: the Top-down Emission Rate Retrieval Algorithm (TERRA) and the Gaussian Dispersion method (GD). Key findings reveal that emission rates ranged from 2,890 to 7,975 m3 \CHfour\ per day, aligning closely with federal estimates. The calculated methane intensities suggest that Canadian offshore production has lower methane intensity compared to onshore production, implying more efficient and potentially safer production practices. This research highlights the importance of accurate emissions measurement to inform regulation and mitigation strategies for offshore oil production.


\subsection*{Paper 3}
\textbf{Title:} Evaluating CO, \NOtwo\, and \SOtwo\ Emissions From Stacks of Turbines and Gas Furnaces of Oil and Gas Processing Complex Using AERMOD \cite{mousavi2022evaluating}.\\
\textbf{Summary:} The research delves into the issue of air pollution originating from industrial sources, specifically the emissions from the stack of turbines and gas furnaces at Maroon oil and gas facilities in Iran. The study aims to fill the knowledge gap in accurately modeling pollutant dispersion using the AERMOD model. After calculating the emission rates, the dispersion was simulated over a 2500 km² area, and the results were verified through field measurements. The study found that pollutant concentrations were higher during the cold season but remained within the limits set by Iranian and US EPA standards. These findings highlight the effectiveness of AERMOD in predicting pollutant dispersion and stress the importance of continuous monitoring and potential mitigation measures, such as the installation of filters and electric compressors, to further reduce emissions and protect public health.


\subsection*{Paper 4}
\textbf{Title:} Dispersion and emission patterns of \NOtwo\ from gas flaring stations in the Niger Delta, Nigeria \cite{nwosisi2020dispersion}.\\
\textbf{Summary:} The research addresses the knowledge gap related to the spatial and temporal patterns of \NOtwo\ emissions from gas flaring stations and their impact on human health and the environment. The study, conducted from January 2017 to December 2018, utilized Aeroqual gas monitors and the HYSPLIT model to measure and forecast \NOtwo\ dispersion from eight gas flaring stations. The findings revealed higher \NOtwo\ concentrations in 2017 compared to 2018, with peak levels during the rainy season. Meteorological factors such as temperature, humidity, and wind speed affected \NOtwo\ concentrations, leading to lower dispersion during low wind speeds. The study underscored the significant spread of \NOtwo\ to non-oil producing states, emphasizing the necessity for protective measures for vulnerable populations.

\subsection*{Paper 5}
\textbf{Title:} Impact of Meteorological Parameters on Dispersion Modeling of Sulfur Dioxide from Gas Flares (Case Study: Sirri Island) \cite{mirrezaei2019impact}.\\
\textbf{Summary:} The paper addresses the knowledge gap in modelling the dispersion of \SOtwo\ from gas flares in coastal areas, particularly on islands with varying meteorological and terrain parameters. The research methodology utilized the CALPUFF dispersion model, a non-steady-state Lagrangian puff model, in conjunction with the Weather Research and Forecasting (WRF) model to simulate the dispersion of \SOtwo\ from flares on Sirri Island in the Persian Gulf over two five-day periods (from
29 January 2011 to 2 February 2011 and 20 October to 24
October 2011). The study incorporated flare parameters based on EPA guidelines into the model. The key findings indicate that low-height flares have a significant impact on ground-level \SOtwo\ concentrations, whereas elevated flares affect areas farther from the flaring site. The results emphasize the importance of considering local meteorological conditions and terrain in air pollution modelling, highlighting the effectiveness of coupling the CALPUFF-WRF models in accurately predicting pollutant dispersion in complex environments.

\subsection*{Paper 6}
\textbf{Title:} Emissions Dispersion Simulated For A Crude Oil Tank Farm Explosion In The Niger Delta Of Nigeria \cite{edeemissions}.\\ 
\textbf{Summary:} This research investigates the environmental and health effects of emissions resulting from a crude oil tank explosion. One significant knowledge gap is the absence of detailed simulation studies on the dispersion of pollutants from unplanned events such as explosions, particularly considering the proximity of oil facilities to populated areas in this region. The study utilizes the AirWare Model alongside the Gaussian plume model to simulate the dispersion of emissions under varying atmospheric conditions. Emission factors, meteorological variables, and site-specific characteristics are employed to assess pollutant concentrations at different distances downwind. The study highlights that pollutants such as suspended particulate matter (SPM), nitrogen oxides (\(\NOx\)), carbon monoxide (CO), and sulphur oxides (\(\SOx\)) notably exceed both national and international air quality standards, especially under unstable atmospheric conditions. These findings emphasize the necessity for stringent regulatory measures and the use of advanced modeling techniques to effectively manage air pollution in the area.

\section*{Summary}
The significance of understanding the dispersion of pollutants from gas flaring and industrial emissions, influenced by meteorological conditions, is well emphasized in the above literature. Advanced dispersion models such as ADMS 5, AERMOD, CALPUFF, and HYSPLIT, along with empirical measurements like the offshore platforms in Newfoundland, are crucial for accurately replicating real-world conditions and forecasting pollutant behaviour.

Research indicates that meteorological factors such as wind speed, temperature, and humidity have a significant impact on the dispersion of pollutants. For instance, in the Niger Delta, pollutants disperse inland during the WAM months and towards both inland and coastal areas in non-WAM months, affecting local communities. A similar pattern was observed in the Sirri Island study, indicating the spread of these pollutants beyond flaring sites, impacting non-oil-producing regions.

Furthermore, the characteristics of pollutants, including the size and heat content of gas flares, play a crucial role. Smaller flares with lower heat content result in higher pollutant concentrations near the source, underscoring the need for mitigation strategies to improve plume buoyancy and reduce local pollution.

In conclusion, continuous monitoring, precise modelling, and targeted mitigation are essential for managing and reducing the adverse impacts of industrial emissions on air quality and public health.